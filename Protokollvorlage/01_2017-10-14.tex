% ggf. neuen Benutzer anlegen

\newPerson{AnE}{ % Vorname 1. Buchstabe, Vorname 2. Buchstabe, Nachname 1. Buchstabe
    Name        = {Anna Example},
    Amt         = {Schatzmeisterin},
    Weblink    = {http://example.org}
}

\begin{Protokoll}{
        Sitzungsleitung                 = {MiL}, 			% Kürzel wie in Hauptdatei definiert
        Protokollfuehrer                = {MaG},  			% Kürzel wie in Hauptdatei definiert
        Sitzungszeit/Datum              = {22.10.2016},   	% \
        Sitzungszeit/BeginnZeit         = {18:30},			%  > Die Verwendung dieser Variablen 
        Sitzungszeit/EndeZeit           = {19:22},			% /   wird im Header konfiguriert!
        Ort                             = {Mumble},			% oder jeder andere Ort
        Aufzeichnung/URL                = {},				% gerne auch leer lassen
        Aufzeichnung/Verantwortlicher   = {MiL},		  	% Kürzel wie in Hauptdatei definiert
        Status                          = {false},			% genehmigt oder nicht (boolean)
    }
    
    \begin{Anwesenheitsliste}
        \anwesenheit{MiL}{1}								% 1, 0, e(ntschuldigt abwesend)
        \anwesenheit{MaG}{1}
        \anwesenheit[Auf der Suche]{AnE}{e}					% in [] Grund angeben oder weglassen (opt. Argument)
    \end{Anwesenheitsliste}
    
    \genehmigeLetztesProtokoll{false}						% boolean
    
    
    \TOP{Tätigkeitsberichte der Vorstandsmitglieder}
        \begin{description}
            \item[\getPersonAsWebsiteLink{MiL}:] \     
            	\begin{itemize}
				        \item Dies und das
				\end{itemize}
            \item[\getPersonAsWebsiteLink{MaG}:]  \     
            	\begin{itemize}
				    \item alles mögliche
				\end{itemize}
		\end{description}
    
    \TOP{Kurzbericht Schatzmeister}
        \begin{center}
            \textbf{Haben:} \\
            \begin{tabular}{|l||r|r|}
                \hline
                \textbf{Girokonto:}            & 12\euro         & (12   \euro)     \\
                \hline
                \textbf{Summe:}                &  12   \euro     & (12 \euro)        \\
                \hline \hline
            \end{tabular} \\
            \vspace{.5em}
            \textbf{Soll:} \\
            \begin{tabular}{|l||r|r|}
                \hline
                 \textbf{Schatzmeister (offenes Budget):}
                &       \euro            & ( \euro)         \\ 
                \textbf{Kredit (bis Jahresende):}    
                &    \euro             & ( \euro)         \\
                \hline
                \textbf{Summe:}                &     \euro         & ( \euro)        \\
                \hline \hline
                \textbf{Kredit XYZ}            &      \euro        & ( \euro) \\
                \hline
            \end{tabular}
        \end{center}
    
    \TOP{Kurzbericht Bezirkssekretär}
        \begin{center}
            \begin{tabular}{|l||r|}
                \hline
                \textbf{Mitglieder gesamt:}             &  12 \\
                \hline
                \hline
            \end{tabular}
        \end{center}
    
    \TOP{Berichte der Beauftragten}
        \begin{description}
            \item[A:] \								% Backslash sorgt dafür, dass die Liste in eine
            										% neue Zeile rutscht; das sieht sauberer aus
            	\begin{enumerate}
            		\item langweilig!
            	
            	\end{enumerate} 
        \end{description}
    
    
    \TOP{Berichte aus den Regionen}
        \begin{description}
            \item[W\"urzburg:] 
            	-- nix los --
            \item[Karlsberg:] 
            	-- noch weniger los --
        \end{description}
    
    
    \TOP{Sonstiges}
    	...
    
    \TOP{Diskussionen und Anträge}
       \Beschluss{
            Umlaufbeschluss     = {false},
            Antragsteller       = {MaG},			% Kürzel wie in Hauptdatei definiert
            Thema               = {Mehr Kaffee!},
            Gesamtwert          = {1.500},			% €-Zeichen wird automatisch ergänzt
            Antragstext         = {Wir möchten mehr Kaffee!},
            Begruendung         = {\
            	\begin{itemize}
            		\item Kaffee ist gesund.
            		\item Kaffe schmeckt gut.
            	\end{itemize}
        	},
            Verantwortlicher    = {MiL},			% Kürzel wie in Hauptdatei definiert
            Abstimmung          = {MaG/1, MiL/e, AnE/x},
            										% Kürzel wie in Hauptdatei definiert;
            										% Werte: 1, 0, e(nthaltung), a(bwesend), 
            										%        x (noch nicht abgestimmt)
            Endergebnis         = {x}, 				% Werte: 1, 0, z(urückgezogen), 
            										%        x (in Abstimmung)
            Bemerkungen         = {}
        }
\end{Protokoll}

